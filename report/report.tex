\documentclass[nocopyrightspace,11pt,authoryear,preprint]{sigplanconf}

\usepackage{amsmath}
\usepackage{listings}

\begin{document}
%% To import in the preambule
%\usepackage{listings}

% "define" SLAC
\lstdefinelanguage{slac}{
  alsoletter={@},
  morekeywords={abstract, case, catch, class, method, do, else, extends, false, final, finally, for, if, implicit, import, match, new, null, object, 
override, package, private, protected, requires, return, sealed, super, self, throw, trait, try, true, type, val, var, while, with, yield, domain, 
postcondition, precondition, invariant, constraint, assert, forAll, in, _, return, @generator, ensure, require, ensuring },
  sensitive=true,
  morecomment=[l]{//},
  morecomment=[s]{/*}{*/},
  morestring=[b]"
}

\newcommand{\codestyle}{\small\sffamily}


% Default settings for code listings
\lstset{
  frame=tb,
  language=slac,
%  aboveskip=3mm,
%  belowskip=3mm,
%  lineskip=-0.1em,
  showstringspaces=false,
  columns=fullflexible,
  mathescape=true,
  numbers=none,
  numberstyle=\tiny,
  basicstyle=\codestyle
} 


\title{Generics Programming in SLAC}
\subtitle{Compiler Construction 2016 Final Report}

\authorinfo{Michael Meyer\and Trace Powers}
           {KTH Royal Institute of Technology}
           {\{meye\}@kth.se}

\maketitle

\section{Introduction}

%Describe in a few words what you did in the first part of the compiler
%project (the non-optional labs), and briefly say what problem you want
%to solve with your extension.

%This section should convince us that you have a clear picture of the
%general architecture of your compiler and that you understand how your
%extension fits in it.

In the required parts of the compiler project, our task was to implement the 
entire pipeline of lexing, parsing, name analysis, type checking, and code generation.

Each of these specific phases transforms the current representation of the 


\section{Examples}

Give code examples where your extension is useful, and describe how
they work with it. Make sure you include examples where the most
intricate features of your extension are used, so that we have an
immediate understanding of what the challenges are.

You can pretty-print code like this:
%\begin{lstlisting}
%class A {
%  method foo(i: Int): Int = {
%    var j: Int;
%    if (i < 0) { j = 0 - i } else { j = i };
%    j + 1
%  }
%}

%method main(): Unit = {
%  println(strOf(new A().foo(0-41)))
%}
%\end{lstlisting}
With generics we can get started into basic data structures.
First, we can implement Linked Lists.

\begin{lstlisting}
class LinkedList[T1] {
  var value : T1;
  var link  : LinkedList[T1];
  method get() : T1 = {
    value
  }
  method set(v : T1) = {
    value = v
  }
  method next() : LinkedList[T1] = {
    link
  }
}
\end{lstlisting}

This allows us to further define things such as Stacks and Queues...

\begin{lstlisting}
class Stack[T1] {
  var top : LinkedList[T1]
  
}
\end{lstlisting}

By using generics, we can provide a richer SLAC standard library, making
the language much more powerful.

%This section should convince us that you understand how your extension can be
%useful and that you thought about the corner cases.

\section{Implementation}

This is a very important section, you explain to us how you made it
work.

Describe all non-obvious tricks you used. Tell us what you thought was
hard and why. If it took you time to figure out the solution to a
problem, it probably means it wasn't easy and you should definitely
describe the solution in details here. If you used what you think is a
cool algorithm for some problem, tell us.  Do not however spend time
describing trivial things (we know what a tree traversal is, for
instance). Cite any reference work you used like this
\cite{TigerBook}.

%hard part.
One issue that came up when testing was that multiple casts of the
generic caused the compiler to fail.  Everything worked fine in a single
cast case, but the second cast caused the type checking on the first one to fail.
After debugging this for a few hours, we found that the issue was hidden not
in the type checking or name analysis, but in the Generic phase of the compiler.
This was because the tree cloning process was sharing some Identifier tree nodes.
This caused the setSymbol command to overwrite each other.
%debugging -  there were shared identifiers thus overwriting each other., values vs vars.

%After reading this section, we should be convinced that you knew what
%you were doing when you wrote your extension, and that you put some
%extra consideration for the harder parts.

\section{Possible Extensions}

%If you did not finish what you had planned, explain here what's
%missing.

%In any case, describe how you could further extend your compiler in
%the direction you chose. This section should convince us that you
%understand the challenges of writing a good compiler for high-level
%programming languages.

%IF DID NOT FINISH:
%Implement hierarchical extensions of generics.

This system only supports one type generic classes. This makes it more difficult
to implement data structures such as maps, although there probably is a special
way to implement them using structures such as tuples.

However, this is still an inconvience for the programmer
and an arbitrary amount of generic types should be supported in the future.
Since generics are just one solution for the general issue of code reusability,
certain features can be tacked on to a basic generics program.
In certain cases, most code of a class can be represented with generics, except
for a single function, such as a library call.  It would still be useful to have
generics for this situation, but it can't be done using the type replacement
mechansism.  By implementing first class functions and then passing a function in
to the type generator as well, this issue could be alieviated.  Of course,
a wrapper class could be written instead but this is what the generic is trying
to replace.


\bibliographystyle{abbrvnat}
\bibliography{bibliography.bib}

\end{document}
