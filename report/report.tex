\documentclass[nocopyrightspace,11pt,authoryear,preprint]{sigplanconf}

\usepackage{amsmath}
\usepackage{listings}

\begin{document}
%% To import in the preambule
%\usepackage{listings}

% "define" SLAC
\lstdefinelanguage{slac}{
  alsoletter={@},
  morekeywords={abstract, case, catch, class, method, do, else, extends, false, final, finally, for, if, implicit, import, match, new, null, object, 
override, package, private, protected, requires, return, sealed, super, self, throw, trait, try, true, type, val, var, while, with, yield, domain, 
postcondition, precondition, invariant, constraint, assert, forAll, in, _, return, @generator, ensure, require, ensuring },
  sensitive=true,
  morecomment=[l]{//},
  morecomment=[s]{/*}{*/},
  morestring=[b]"
}

\newcommand{\codestyle}{\small\sffamily}


% Default settings for code listings
\lstset{
  frame=tb,
  language=slac,
%  aboveskip=3mm,
%  belowskip=3mm,
%  lineskip=-0.1em,
  showstringspaces=false,
  columns=fullflexible,
  mathescape=true,
  numbers=none,
  numberstyle=\tiny,
  basicstyle=\codestyle
} 


\title{Generics Programming in SLAC}
\subtitle{Compiler Construction 2016 Final Report}

\authorinfo{Michael Meyer\and Trace Powers}
           {KTH Royal Institute of Technology}
           {\{meye,trace\}@kth.se}

\maketitle

\section{Introduction}

%Describe in a few words what you did in the first part of the compiler
%project (the non-optional labs), and briefly say what problem you want
%to solve with your extension.

%This section should convince us that you have a clear picture of the
%general architecture of your compiler and that you understand how your
%extension fits in it.

In the required parts of the project, our task was to implement the 
entire pipeline of the compiler, which includes 
lexing, parsing, name analysis, type checking, and code generation.

Each of these specific phases transforms the current representation of the 
program into a more useful representation of the desired program, along
with catching any errors that come along the way.

First, lexing converts plain text into a sequence of tokens.  This 
is useful because we have an understanding of the meaning of a token as
a part of the language.  a '=' in plain text isn't meaningful because
we don't know if it is part of an equals sign or an assignment.  However,
an EQSIGN or an EQUALS is much more meaningful.

Parsing converts the sequence of tokens into a tree.  This tree is the
organization of the program.  It describes not just what is in the program, such
as classes and identifiers, but rather how the parts come
together to form a more complex meaning.

Name analysis attaches the concept of a specific variable to many identifiers so
the compiler then understands what the identifier 'x' means in a specific context.

Type checking and code generation don't necessarily transform the representation of the
program, but are needed to produce the end result of the binary.

Code reuse is an important feature of a language.  Classes allow us 
to reuse many functions through inheritance.  However, many clases 
might have to be written multiple times to support different types.
Interfaces solve this problem, but in doing so produce unneeded code.
Generics allow us to sidestep this issue by generating new versions of a general 
class for the specific type we need on the fly automatically in the compiler.
This project aims to implement generics support for the SLAC programming
language.

\section{Examples}

The biggest use of generics is in data structures and algorithms,
where the data structure acts the same for all types.

Generics allow for the creation of a standard library, which
can have all the useful data structures already implemented 
for the user.

The first data structure we looked into was the linked list.

%Give code examples where your extension is useful, and describe how
%they work with it. Make sure you include examples where the most
%intricate features of your extension are used, so that we have an
%immediate understanding of what the challenges are.

\begin{lstlisting}
class List[T] {

  var val : T;
  var next : List[T];
  var hasNext : Bool;

  method init(v : T): List[T] = {
    val = v;
    hasNext = false;
    self 
  }

  method append(v : T): Unit = {
    if(hasNext) {
      next.append(v)
    } else {
      next = new List[T]().init(v);
      hasNext = true
    }
  }

  method attach(v : T): List[T] = {
    var tmp : List[T];
    tmp = new List[T]().init(v);
    tmp.set_next(self);
    tmp 
  }

  method set_next(v : List[T]) : Unit = {
    next = v;
    hasNext = true
  }


  method set(n : Int, v : T) : Unit = {
    if(n == 0) {
      val = v
    } else {
      next.set(n-1,v)
    }
  }

  method get(n : Int) : T = {
    if(n == 0) {
      val
    } else {
      next.get(n-1)
    }
  }
  
  method next_pointer() : List[T] = {
    next
  }

  method len(): Int = {
    if(hasNext) {
      1 + next.len()
    } else {
      1
    }
  }

}

\end{lstlisting}

The code above creates a list class for any type.
It supports appending to the list, getting and setting objects and
getting the length.  The user can then use this in their own methods 
shown below.

\begin{lstlisting}
var m : List[Int];
m = new List[Int].init(5);
m.append(6);
m.append(104);
m.get(2)
\end{lstlisting}

With a list class defined for all objects, more complex data
structures such as Stacks and Queues can be formed.  They
are defined below.

\begin{lstlisting}
class Stack[T] {
  
  var lst : List[T];

  method init(v : T): Stack[T] = {
    lst = new List[T]().init(v);
    self
  }

  method push(v : T): Unit = {
    lst = lst.attach(v)
  }

  method pop(): T = {
    var tmp : T;
    tmp = lst.get(0);
    lst = lst.next_pointer();
    tmp
  }

}


class Queue[T] {

  var lst : List[T];

  method init(v : T): Queue[T] = {
    lst = new List[T]().init(v);
    self 
  }
  method push(v : T): Unit = {
    lst.append(v)
  }
  method pop(): T = {
    var tmp: T;
    tmp = lst.get(0);
    lst = lst.next_pointer();
    tmp
  }

}
\end{lstlisting}

The stacks and queues can be run with the code below (results also shown).
\begin{lstlisting}
stk = new Stack[Int]().init(10);
stk.push(20);
stk.push(30);
que = new Queue[Int]().init(10);
que.push(20);
que.push(30);
 
c = 3;
while(0 < c) {
  println("ITER");
  println("stack" + strOf(stk.pop()));
  println("queue" + strOf(que.pop()));
  c = c - 1
};

==== RESULTS ====
ITER
stack30
queue10
ITER
stack20
queue20
ITER
stack10
queue30
\end{lstlisting}

Furthermore, functional programming types can be constructed, such as the Option type.

\begin{lstlisting}
class Option[T] {
  var value : T;
  var has : Bool;

  method some(v : T): Option[T] = {
    value = v;
    has = true;
    self
  }
  method none(): Option[T] = {
    has = false;
    self
  }
  method get(): T = {
    value
  }
  method contains(): Bool = {
    has
  }
}
\end{lstlisting}

\section{Implementation}

%This is a very important section, you explain to us how you made it
%work.

%Describe all non-obvious tricks you used. Tell us what you thought was
%hard and why. If it took you time to figure out the solution to a
%problem, it probably means it wasn't easy and you should definitely
%describe the solution in details here. If you used what you think is a
%cool algorithm for some problem, tell us.  Do not however spend time
%describing trivial things (we know what a tree traversal is, for
%instance). Cite any reference work you used like this
%\cite{TigerBook}.

We implemented generics support by adding another stage to the pipeline.
This stage is called the generics stage and fits in between the parsing stage
and the name analysis, both of which had to be slightly modified, mostly
to recognize identifiers that can have brackets next to them (aka List[T] or
List[Int]).

The generics phase scans the entire program and searches for generic uses,
based off variable delcarations and new object creation.  It obtains a list
for each generic type of which concrete versions to generate on the fly.

For example, if a List[Int] was called in the main method, the compiler
would know to 'implement' the generic List class for type Int.

To 'implement' a class, it clones the classDecl tree and replaces any 
identifiers of type T ( or whatever the user called the generic type ),
with the identifier or type of the concrete version, aka the Int.

Then all references of List[Int] are changed to the concrete class, which
is named with a '\#' to the end of the class. So the concrete List[Int] is
converted into a List\#Int.  Then the type checker can just look up the 
List\#Int class as normal.

%hard part.
The first big issue that came up when testing was that multiple casts of the
generic caused the compiler to fail.  Everything worked fine in a single
cast case, but the second cast caused the type checking on the first one to fail.
After debugging this for a few hours, we found that the issue was hidden not
in the type checking or name analysis, but in the generic phase of the compiler.
This was because the tree cloning process was sharing some Identifier tree nodes.
This caused the setSymbol command to overwrite each other, thus making a String cast,
which required a symbol with type String, to instead get the symbol for type Int.

Another issue that came up was the generic type dependencies.  For example,
if a Stack[T] requires a List[T] for the implementation, but a concrete List[T]
is never created somewhere else in the program,
the compiler originally failed because the Stack[T] called a class that didn't exist.
To solve this, we had to use a queue to specify a work list of tuples of type
(ClassDecl, Type).  The compiler would generate this generic class of type Type, and then
also add to the queue any types it depends on.  This solved the issue because then
since Stack[Int] requires a List[Int], the list would be created even though the user
never wrote List[Int] in their code.

\section{Possible Extensions}

%If you did not finish what you had planned, explain here what's
%missing.

%In any case, describe how you could further extend your compiler in
%the direction you chose. This section should convince us that you
%understand the challenges of writing a good compiler for high-level
%programming languages.

%IF DID NOT FINISH:
%Implement hierarchical extensions of generics.

This system only supports one type generic classes. This makes it more difficult
to implement data structures such as maps, although there probably is a special
way to implement them using structures such as tuples.

However, this is still an inconvience for the programmer
and an arbitrary amount of generic types should be supported in the future.
Since generics are just one solution for the general issue of code reusability,
certain features can be tacked on to a basic generics program.
In certain cases, most code of a class can be represented with generics, except
for a single function, such as a specific library call.  It would still be useful to have
generics for this situation, but it can't be done using the type replacement
mechansism.  By implementing first class functions and then passing a function in
to the type generator as well, this issue could be alieviated.  Of course,
a wrapper class could be written instead but this is uneeded code and 
what the generic is trying to replace.

\bibliographystyle{abbrvnat}
\bibliography{bibliography.bib}

\end{document}
